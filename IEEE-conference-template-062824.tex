\documentclass[conference]{IEEEtran}
\IEEEoverridecommandlockouts
% The preceding line is only needed to identify funding in the first footnote. If that is unneeded, please comment it out.
%Template version as of 6/27/2024

\usepackage{cite}
\usepackage{amsmath,amssymb,amsfonts}
\usepackage{algorithmic}
\usepackage{graphicx}
\usepackage{textcomp}
\usepackage{xcolor}

\def\BibTeX{{\rm B\kern-.05em{\sc i\kern-.025em b}\kern-.08em
    T\kern-.1667em\lower.7ex\hbox{E}\kern-.125emX}}
\begin{document}

\title{An Intelligent Rollator by Integrating Fuzzy Control and Multi-Sensor System\\
\thanks{Identify applicable funding agency here. If none, delete this.}

}

\author{
	\IEEEauthorblockN{Chen, Hsiang-Chieh}
	\IEEEauthorblockA{
		\textit{Department of Mechanical Engineering} \\
		\textit{National Central University}\\
		Taoyuan, Taiwan \\
		hcchen@cc.ncu.edu.tw
	}
	\and

	\IEEEauthorblockN{2}
	\IEEEauthorblockA{
		\textit{Department of Mechanical Engineering} \\
		\textit{National Central University}\\
		Taoyuan, Taiwan \\
		@cc.ncu.edu.tw
	}
	\and

	\IEEEauthorblockN{Lin, Yi-Hsuan}
	\IEEEauthorblockA{
		\textit{Department of Mechanical Engineering} \\
		\textit{National Central University}\\
		Taoyuan, Taiwan \\
		113323090@cc.ncu.edu.tw
	}
	\and

	\IEEEauthorblockN{4}
	\IEEEauthorblockA{
		\textit{Department of Mechanical Engineering} \\
		\textit{National Central University}\\
		Taoyuan, Taiwan \\
		@cc.ncu.edu.tw
	}
	\and
	
	\IEEEauthorblockN{5}
	\IEEEauthorblockA{
		\textit{Department of Mechanical Engineering} \\
		\textit{National Central University}\\
		Taoyuan, Taiwan \\
		@cc.ncu.edu.tw
	}
	\and
	
	\IEEEauthorblockN{6}
	\IEEEauthorblockA{
		\textit{Department of Mechanical Engineering} \\
		\textit{National Central University}\\
		Taoyuan, Taiwan \\
		@cc.ncu.edu.tw
	}
	\and
	
	\IEEEauthorblockN{7}
	\IEEEauthorblockA{
		\textit{Department of Mechanical Engineering} \\
		\textit{National Central University}\\
		Taoyuan, Taiwan \\
		@cc.ncu.edu.tw
	}
	\and
}

\maketitle

\begin{abstract}
This research presents an innovative mobility aid solution, termed the ``Intelligent Rollator'', designed for elderly individuals with mobility limitations. 
The system utilizes Light Detection and Ranging (LiDAR) to acquire the user's position, and a nine-axis sensor to monitor rollator’s status. 
To drive the rollator and assist the user, two hub motors are integrated into the rollator. 
To achieve synchronized operation and precise control the system continuously monitors the Hall signals from both motors and calculates the speed differential.
By integrating data from sensors, the system analyze of the user's walking status, and adapt to various conditions.
To control the motors with data from the sensors, a two-stage hierarchical fuzzy logic controller is designed. 
The first stage generating initial control signals based on the user's status {relative to the rollator}. 
The second stage takes the first stage output, combined with the calculated motor speed differential, to fin-tune the initial control signals and achieving more precise motor control. 
This design ensures the Intelligent Rollator can adjust motor speeds in real-time, maintaining a stable movement, which provides reliable mobility support.
\end{abstract}

\begin{IEEEkeywords}
Intelligent rollator, LiDAR, Fuzzy logic system, Mobility control.
\end{IEEEkeywords}

\section{Introduction}
This document is a model and instructions for \LaTeX.
Please observe the conference page limits. For more information about how to become an IEEE Conference author or how to write your paper, please visit   IEEE Conference Author Center website: https://conferences.ieeeauthorcenter.ieee.org/.

\section{System Overview}
LiDAR is used to detect the variation of the positions of both legs in the forward direction of the senior while walking, as well as the rate of variation in the spacing of the gait in order to deduce the walking trend.
The LiDAR scanner scans frontward, while the angular scanning regions for the left and right legs are set to 0° to 30° and 330° to 360°, respectively. For normal usage of the user and assistive walker, the safe distance is about 80 cm. If the scanned distance exceeds this value, then an abnormal situation is detected, and this result is used to adjust the speed of motor rotation.
Finally, after obtaining the sensing data, bubble sort is used to arrange the data and eliminate any abnormal points, and then the mean value is calculated. The system calculates a new center point of the leg positions every two seconds and derives a weighted mean value between the newly calculated center point and the mean value. If the value of the center distance is below a specific range, the motor speed will be increased; otherwise, it will be decreased.
For further analysis of the walking patterns, smoothing functions are applied to the average gait spacing information in order to eliminate noise and short-term variations. Following that, the peak and trough information are obtained from the smoothed data set, and the Least Squares Method is applied for curve fitting to determine the best fit model for the trend. Finally, the probability density function (PDF), that of a Gaussian distribution, is applied for statistical analysis of the gait information, and the resulting distribution will be output for analysis of the walking stability and abnormality.


\section{Proposed Methodology}
This section presents the dual in-wheel motor control and synchronization strategy for the intelligent walker. The left and right wheel outputs are adjusted according to the distance between the elderly user and the walker to maintain a safe following distance and stable walking.
A brushless motor driver is used. The control wiring includes Hall-signal feedback, PWM speed control, brake power-off, three-speed selection, and a reverse signal. Hall signals are fed back to an Arduino Mega2560 for closed-loop control, PWM directly regulates the two motor speeds, and the brake power-off line is used for emergency stopping; the reverse signal can provide downhill resistance but may induce oscillation.
Even motors of the same model show left-right speed differences due to manufacturing tolerances, so feedback-based synchronization is required. Feedback sources include Hall signals and the absolute heading deviation and angular acceleration from a 9-axis sensor to improve straight-line and turning stability.
First, hall readings change abruptly during N/S transitions and are used to identify speed pulses. Under the same PWM input, the left and right wheels show different transition counts (e.g., 20 vs. 16), and the waveforms differ across PWM ranges (60–75 and 80–90), indicating persistent speed mismatch.Besides, turning control uses a 9-axis sensor to estimate posture and operator intent, and differential-speed control is applied. Tuning results set the increment/decrement to ±7, with the inner-side speed reduced and the outer-side speed increased during turns. The BNO055 sensor provides absolute orientation, angular velocity, acceleration, and magnetic-field strength as control inputs.
The controller adopts a fuzzy logic system (FLS) and a hierarchical fuzzy system (HFS). The FLS consists of fuzzification, inference, and defuzzification, while the HFS is composed of multiple FLS units. Using the Nauck index, the complexity is minimized and the rule number is reduced when each layer has two inputs. The HFS is implemented in serial (SHFS) or parallel (PHFS) forms.
In the dual-wheel fuzzy walking system, the left-wheel controller uses the operator-LiDAR distance and its rate of change to output the left-wheel speed difference. The right-wheel controller incorporates the left-wheel output and the Hall-recorded speed difference within a fixed interval to estimate and adjust the right-wheel speed difference for synchronization. 
Finally, synchronization is evaluated by the Hall-signal difference: the initial speed gap is larger but converges over time, and the steady-state error stays within approximately ±2.


\section{Experiment and Conclusion}
In the experimental phase of this study, raw distance data were first pre-processed to mitigate the impact of outliers. 

Subsequently, the least squares method was applied to derive an optimal fitting curve. 
This curve accurately characterizes the spatial variation between the user's lower limbs and the LiDAR sensor, enabling the system to distinguish dynamic features between steady-state walking and gait transitions (e.g., acceleration or deceleration). 

These features serve as the primary inputs for the fuzzy control system, thereby facilitating precise regulation of the in-wheel motors.

The performance of the proposed Hierarchical Fuzzy System (HFS) was validated across diverse 
terrain scenarios. Experimental results indicate that in straight-line tests, the system 
effectively constrains the wheel speed differential within a narrow interval of $\pm 2$ (as 
illustrated in Fig.~\ref{fig:fig5_1}), ensuring autonomous directional stability without human intervention. 
In cornering maneuvers, the controller exhibits high sensitivity to user intent; 
upon detecting a significant speed differential, it guides the vehicle 
through the turn and achieves rapid convergence back to straight-line tracking post-maneuver. 

Furthermore, the system demonstrates robust adaptability in ramp scenarios. 
For uphill transitions, the system dynamically compensates the Pulse Width Modulation (PWM) output based on the relative distance between the user and the vehicle's center of mass to provide sufficient propulsion. 
Conversely, during downhill descents, an automatic power-off braking mechanism is implemented to simulate the functionality of an Anti-lock Braking System (ABS), providing sustained braking torque and ensuring operational safety.

\begin{figure}[h]
    \centering
    \includegraphics[width=\linewidth]{fig5_1.png}
    \caption{Hall signal difference during straight-line walking and turning maneuvers. The green boundary illustrates the system's ability to maintain the speed differential within a stable range of $\pm 2$ during the correction process.}
    \label{fig:fig5_1}
\end{figure}

\begin{thebibliography}{00}
\bibitem{b1} G. Eason, B. Noble, and I. N. Sneddon, ``On certain integrals of Lipschitz-Hankel type involving products of Bessel functions,'' Phil. Trans. Roy. Soc. London, vol. A247, pp. 529--551, April 1955.
\bibitem{b2} J. Clerk Maxwell, A Treatise on Electricity and Magnetism, 3rd ed., vol. 2. Oxford: Clarendon, 1892, pp.68--73.
\bibitem{b3} I. S. Jacobs and C. P. Bean, ``Fine particles, thin films and exchange anisotropy,'' in Magnetism, vol. III, G. T. Rado and H. Suhl, Eds. New York: Academic, 1963, pp. 271--350.
\bibitem{b4} K. Elissa, ``Title of paper if known,'' unpublished.
\bibitem{b5} R. Nicole, ``Title of paper with only first word capitalized,'' J. Name Stand. Abbrev., in press.
\bibitem{b6} Y. Yorozu, M. Hirano, K. Oka, and Y. Tagawa, ``Electron spectroscopy studies on magneto-optical media and plastic substrate interface,'' IEEE Transl. J. Magn. Japan, vol. 2, pp. 740--741, August 1987 [Digests 9th Annual Conf. Magnetics Japan, p. 301, 1982].
\end{thebibliography}

\end{document}
