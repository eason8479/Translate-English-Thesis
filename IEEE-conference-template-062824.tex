\documentclass[conference]{IEEEtran}
\IEEEoverridecommandlockouts
% The preceding line is only needed to identify funding in the first footnote. If that is unneeded, please comment it out.
%Template version as of 6/27/2024

\usepackage{cite}
\usepackage{amsmath,amssymb,amsfonts}
\usepackage{algorithmic}
\usepackage{graphicx}
\usepackage{textcomp}
\usepackage{xcolor}

\def\BibTeX{{\rm B\kern-.05em{\sc i\kern-.025em b}\kern-.08em
    T\kern-.1667em\lower.7ex\hbox{E}\kern-.125emX}}
\begin{document}

\title{An Intelligent Rollator by Integrating Fuzzy Control and Multi-Sensor System}

\author{
	\IEEEauthorblockN{Chen, Hsiang-Chieh}
	\IEEEauthorblockA{
		\textit{Department of Mechanical Engineering} \\
		\textit{National Central University}\\
		Taoyuan, Taiwan \\
		hcchen@cc.ncu.edu.tw
	}
	\and

	\IEEEauthorblockN{2}
	\IEEEauthorblockA{
		\textit{Department of Mechanical Engineering} \\
		\textit{National Central University}\\
		Taoyuan, Taiwan \\
		@cc.ncu.edu.tw
	}
	\and

	\IEEEauthorblockN{Lin, Yi-Hsuan}
	\IEEEauthorblockA{
		\textit{Department of Mechanical Engineering} \\
		\textit{National Central University}\\
		Taoyuan, Taiwan \\
		113323090@cc.ncu.edu.tw
	}
	\and

	\IEEEauthorblockN{Lu, Yu-Ta}
	\IEEEauthorblockA{
		\textit{Department of Mechanical Engineering} \\
		\textit{National Central University}\\
		Taoyuan, Taiwan \\
		114323084@cc.ncu.edu.tw
	}
	\and
	
	\IEEEauthorblockN{Lin, Yi-Jie}
	\IEEEauthorblockA{
		\textit{Department of Mechanical Engineering} \\
		\textit{National Central University}\\
		Taoyuan, Taiwan \\
		111303003@cc.ncu.edu.tw
	}
	\and
	
	\IEEEauthorblockN{Chen, Ting-You}
	\IEEEauthorblockA{
		\textit{Department of Mechanical Engineering} \\
		\textit{National Central University}\\
		Taoyuan, Taiwan \\
		114323102@cc.ncu.edu.tw
	}
	\and
	
	\IEEEauthorblockN{7}
	\IEEEauthorblockA{
		\textit{Department of Mechanical Engineering} \\
		\textit{National Central University}\\
		Taoyuan, Taiwan \\
		@cc.ncu.edu.tw
	}
	\and
}

\maketitle

\begin{abstract}
This research presents an innovative mobility aid solution, termed the ``Intelligent Rollator'', designed for elderly individuals with mobility limitations. 
The system utilizes Light Detection and Ranging (LiDAR) to acquire the user's position, and a nine-axis sensor to monitor rollator’s status. 
To drive the rollator and assist the user, two hub motors are integrated into the rollator. 
To achieve synchronized operation and precise control the system continuously monitors the Hall signals from both motors and calculates the speed differential.
By integrating data from sensors, the system analyze of the user's walking status, and adapt to various conditions.
To control the motors with data from the sensors, a two-stage hierarchical fuzzy logic controller is designed. 
The first stage generating initial control signals based on the user's status {relative to the rollator}. 
The second stage takes the first stage output, combined with the calculated motor speed differential, to fin-tune the initial control signals and achieving more precise motor control. 
This design ensures the Intelligent Rollator can adjust motor speeds in real-time, maintaining a stable movement, which provides reliable mobility support.
\end{abstract}

\begin{IEEEkeywords}
Intelligent rollator, LiDAR, Fuzzy logic system, Mobility control.
\end{IEEEkeywords}

\section {Introduction}
Based on the population projection by National Development Council (NDC) \cite{b1}, the proportion of elder population in Taiwan is growing rapidly, and will only go faster in as time goes on in the foreseeable future.

In this research, we design the ``Intelligent Rollator'', which can give solid assistance to those who can stand, and walk, but still need extra support. The rollator itself is shown in Fig.~\ref{fig:fig_1}

\begin{figure}
    \centering
    \includegraphics[width = 5 cm]{fig_2.jpg}
    \caption{The side view of intelligent rollator}
    \label{fig:fig_1}
\end{figure}

\section {Proposed Methodology}

\subsection {Detect User Walking Behaviors}
Different people have different walking behaviors, including cadence, step length, etc. 
So, clearly identify variance between different user is crutial for the rollator. Otherwise, it can't give stable and useful assistance to the users.

To address this issue, we use Light Detection and Ranging (LiDAR) to detect both legs position. 
The angular scanning regions are set to left region 0° to 30° for left leg, and right region 330° to 360° for the right leg. 
By restricting the scanning area, we reduced the noise from LiDAR, thus getting higher-quality data.

After obtaining the sensing data, we use bubble sort arrange the data and eliminate any abnormal points. 
Then the mean value from both regions are calculated, which is the center point each leg position. 
The position calculates and update cycle is 2 Hz.

By grouping all the data from LiDAR, we can have the time domain data about current user walking behaviors.
Through smoothing function, peak and trough finding algotherm, and statistical analysis, we extract the more useful data from the initial time domain data, which will be use in fuzzy control \cite{b2}.

\subsection {Motor Control}
In this study, we use dual in-wheel motors to power rollator, and use synchronization strategy to control it. 
The left and right wheel outputs are adjusted according to the distance between the elderly user and the walker to maintain a safe following distance and stable walking.

Similar to humans, individual motors also have variance between each others. 
To mantain system stability, and user safty, we designed a close-loop control in which an Arduino Mega2560 utilizes feedback from Hall effect sensors, and data from a 9-axis sensor to improve straight-line and turning stability.

By applying different signal through different controll wires, we can control motors behavior.
An important behavior we notice in research is that when applying rever signal for break, the rollator will vibrate violently. 
So we didn't use it for breaking, mainly to keep the users safe.

\subsection {Fuzzy Control System}
The controller adopts a fuzzy logic system (FLS) and a hierarchical fuzzy system (HFS) \cite{b3}. The FLS consists of fuzzification, inference, and defuzzification, while the HFS is composed of multiple FLS units. 

Using the Nauck index \cite{b4}, the complexity is minimized and the rule number is reduced when each layer has two inputs. The HFS is implemented in serial (SHFS) or parallel (PHFS) forms.

In the dual-wheel fuzzy walking system, the left-wheel controller uses the operator-LiDAR distance and its rate of change to output the left-wheel speed difference. Which is shown in Fig.\ref{fig:fig_3}

\begin{figure}[h]
    \centering
    \includegraphics[width=\linewidth]{fig_3.png}
    \caption{Left wheel fuzzy control logic}
    \label{fig:fig_3}
\end{figure}

The right-wheel controller incorporates the left-wheel output and the Hall-recorded speed difference within a fixed interval to estimate and adjust the right-wheel speed difference for synchronization.

\section{Experiment and Conclusion}
The performance of the proposed Hierarchical Fuzzy System (HFS) was validated across diverse scenarios. 
Experimental results indicate that in straight-line tests, the system effectively constrains the wheel speed differential within a narrow interval of $\pm 2$ (as illustrated in Fig.~\ref{fig:fig5_1}), ensuring autonomous directional stability without human intervention.

In turning test, the controller is sensitive to user intent. Once it detect a significant speed differential, it guides the vehicle through the turn and switch back to straight-line tracking once the corner is passed. 

For uphill, the rollator did provide sufficient propulsion through PWN to assist the user, which assit them pass a more challenging terrian.
As for downhill, power-off braking providing sustained braking torque and ensuring operational safety.

In conculsion, this design can assiste the user in many daily use cases. 
We hope it can assist elders and weaks, so they can exercise more, and have better life quality.


\begin{figure}[h]
    \centering
    \includegraphics[width=\linewidth]{fig5_1.png}
    \caption{Hall signal difference during straight-line walking and turning maneuvers. 
    The green boundary illustrates the system's ability to maintain the speed differential within a stable range of $\pm 2$ during the correction process.}
    \label{fig:fig5_1}
\end{figure}

\begin{thebibliography}{00}
\bibitem{b1} National Development Council [Online]. Available: https://pop-proj.ndc.gov.tw/main\_en/
\bibitem{b2} A. Yeaser, J. Tung, J. Huissoon, and E. Hashemi, ``Learning-Aided User Intent Estimation for Smart Rollators,'' \textit{2020 42nd Annual International Conference of the IEEE Engineering in Medicine \& Biology Society (EMBC)}, Montreal, QC, Canada, 2020, pp. 3178--3183, doi: 10.1109/EMBC44109.2020.9175610.

\bibitem{b3} T. Zhao, H. Cao, and S. Dian, ``A Self-Organized Method for a Hierarchical Fuzzy Logic System Based on a Fuzzy Autoencoder,'' \textit{IEEE Transactions on Fuzzy Systems}, vol. 30, no. 12, pp. 5104--5115, Dec. 2022, doi: 10.1109/TFUZZ.2022.3165690.

\bibitem{b4} D. D. Nauck, ``Measuring Interpretability in Rule-Based Classification Systems,'' \textit{The 12th IEEE International Conference on Fuzzy Systems, 2003. FUZZ '03.}, St. Louis, MO, USA, 2003, pp. 196--201 vol.1, doi: 10.1109/FUZZ.2003.1209361.

\end{thebibliography}

\end{document}